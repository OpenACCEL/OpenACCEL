\section{Global} 
\subsection{Members} 
\subsubsection{inBrowser} 
\begin{description} 
\item[Description:]We have two environments: the browser and Node. If we are in the browser we should use requirejs and call our functions that way.
If we are in node, we should not and instead should just export the various modules for testing purpose.
\end{description} 
\subsection{Methods} 
\section{TemplateClass} 
\textbf{Description: }Classes can be defined as objects. Indiciate this using the @class param.
\subsection{Members} 
\subsection{Methods} 
\subsubsection{compile(sweet)} 
\paragraph{Description:} \hfill \\ 
Main and testable compilation function.
\paragraph{Parameters:} \hfill \\ 
\begin{tabular}{|l|l|l|}
\hline
\textbf{Name} & \textbf{Type} & \textbf{Description} \\ 
\hline
\texttt{sweet} &  & A possible reference to the sweet library.\\ 
\hline
\end{tabular}
\paragraph{Returns:} \hfill \\ 
Returns the number 6 if compilation and execution of the code with Sweet.js macro has all been succesful.\\ 
