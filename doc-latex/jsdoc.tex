\section{Global} 
\subsection{Members} 
\subsubsection{inBrowser} 
\begin{description} 
\item[Description:]Browser vs. Node
\end{description} 
\subsubsection{inBrowser} 
\begin{description} 
\item[Description:]Browser vs. Node
\end{description} 
\subsubsection{inBrowser} 
\begin{description} 
\item[Description:]Browser vs. Node
\end{description} 
\subsubsection{inBrowser} 
\begin{description} 
\item[Description:]Browser vs. Node
\end{description} 
\subsection{Methods} 
\section{FuncPass} 
\textbf{Description: }Classes can be defined as objects. Indiciate this using the @class param.
\subsection{Members} 
\subsection{Methods} 
\subsubsection{parse(scriptlines)} 
\paragraph{Description:} \hfill \\ 
Parse the given lines of script.
Should be overridden by subclasses.
The base class only contains precondition checking and returns the given input.
\paragraph{Parameters:} \hfill \\ 
\begin{tabular}{|l|l|l|}
\hline
\textbf{Name} & \textbf{Type} & \textbf{Description} \\ 
\hline
\texttt{scriptlines} & String[] & Lines of script that need to be parsed.\\ 
\hline
\end{tabular}
\paragraph{Preconditions:} 
\begin{itemize}  
\item  scriptLines != null
\item  scriptLines != undefined
\end{itemize}  
\paragraph{Returns:} \hfill \\ 
the input\\ 
\underline{Type:} String[]
\subsubsection{parse(script)} 
\paragraph{Description:} \hfill \\ 
Parses the input script to a script thar can be expanded by macros.
\paragraph{Parameters:} \hfill \\ 
\begin{tabular}{|l|l|l|}
\hline
\textbf{Name} & \textbf{Type} & \textbf{Description} \\ 
\hline
\texttt{script} & String & input (Accel) scrpit\\ 
\hline
\end{tabular}
\paragraph{Preconditions:} 
\begin{itemize}  
\item  script != null
\item  script != undefined
\end{itemize}  
\paragraph{Returns:} \hfill \\ 
exmpandable script.\\ 
\underline{Type:} String
\subsubsection{scriptToLines(script)} 
\paragraph{Description:} \hfill \\ 
Transforms the input {@code script} to an array of strings,
each representing a line.
\paragraph{Parameters:} \hfill \\ 
\begin{tabular}{|l|l|l|}
\hline
\textbf{Name} & \textbf{Type} & \textbf{Description} \\ 
\hline
\texttt{script} & String & the input script given as a String\\ 
\hline
\end{tabular}
\paragraph{Preconditions:} 
\begin{itemize}  
\item  script != null
\item  script != undefined
\end{itemize}  
\paragraph{Returns:} \hfill \\ 
an array containing all lines in {@code script}\\ 
\underline{Type:} String[]
\subsubsection{translateLine(a)} 
\paragraph{Description:} \hfill \\ 
Translates a line of Accel script to a line that can be expanded using macros.
Examples:
x = 5 becomes func(x = 5)
x = 5 ; kg becomes func(x = 5 ; {'kg' : 1})
z = 2 + sin(y + sin(x)) + 4 + sin(2) becomes func(z = 2 + sin(exe.y() + sin(exe.x())) + 4 + sin(2))
\paragraph{Parameters:} \hfill \\ 
\begin{tabular}{|l|l|l|}
\hline
\textbf{Name} & \textbf{Type} & \textbf{Description} \\ 
\hline
\texttt{a} & String & line of Accell script\\ 
\hline
\end{tabular}
\paragraph{Preconditions:} 
\begin{itemize}  
\item  line != null
\item  line != undefined
\end{itemize}  
\paragraph{Returns:} \hfill \\ 
Translated line\\ 
\underline{Type:} String
\subsubsection{translateRHS(rhs)} 
\paragraph{Description:} \hfill \\ 
Translates the right hand side of an Accel definition to a macro compatible string.
\paragraph{Parameters:} \hfill \\ 
\begin{tabular}{|l|l|l|}
\hline
\textbf{Name} & \textbf{Type} & \textbf{Description} \\ 
\hline
\texttt{rhs} & String & Right hand side of an Accel definitions\\ 
\hline
\end{tabular}
\paragraph{Preconditions:} 
\begin{itemize}  
\item  rhs != null
\item  rhs != undefined
\end{itemize}  
\paragraph{Returns:} \hfill \\ 
a macro compatible string.\\ 
\underline{Type:} String
\subsubsection{translateUnits(units)} 
\paragraph{Description:} \hfill \\ 
Translate the units of a certain line into our format, such
that it is an object in the executable code.
\paragraph{Parameters:} \hfill \\ 
\begin{tabular}{|l|l|l|}
\hline
\textbf{Name} & \textbf{Type} & \textbf{Description} \\ 
\hline
\texttt{units} & String & the units in a String format\\ 
\hline
\end{tabular}
\paragraph{Preconditions:} 
\begin{itemize}  
\item  units != null
\item  units != undefined
\end{itemize}  
\paragraph{Returns:} \hfill \\ 
a String to be used in our executable code.\\ 
\underline{Type:} String
\subsubsection{trimLines(scriptArray)} 
\paragraph{Description:} \hfill \\ 
Trims the lines in {@code scriptArray}.
\paragraph{Parameters:} \hfill \\ 
\begin{tabular}{|l|l|l|}
\hline
\textbf{Name} & \textbf{Type} & \textbf{Description} \\ 
\hline
\texttt{scriptArray} & String[] & the script formatted in a number of lines\\ 
\hline
\end{tabular}
\paragraph{Returns:} \hfill \\ 
the trimmed script in the format of a number of lines\\ 
\underline{Type:} String[]
\subsubsection{compile(sweet)} 
\paragraph{Description:} \hfill \\ 
Main and testable compilation function.
\paragraph{Parameters:} \hfill \\ 
\begin{tabular}{|l|l|l|}
\hline
\textbf{Name} & \textbf{Type} & \textbf{Description} \\ 
\hline
\texttt{sweet} &  & A possible reference to the sweet library.\\ 
\hline
\end{tabular}
\paragraph{Returns:} \hfill \\ 
Returns the number 6 if compilation and execution of the code with Sweet.js macro has all been succesful.\\ 
\section{Pass} 
\textbf{Description: }Base class for passes of the preprocessor.
\subsection{Members} 
\subsection{Methods} 
\subsubsection{parse(scriptlines)} 
\paragraph{Description:} \hfill \\ 
Parse the given lines of script.
Should be overridden by subclasses.
The base class only contains precondition checking and returns the given input.
\paragraph{Parameters:} \hfill \\ 
\begin{tabular}{|l|l|l|}
\hline
\textbf{Name} & \textbf{Type} & \textbf{Description} \\ 
\hline
\texttt{scriptlines} & String[] & Lines of script that need to be parsed.\\ 
\hline
\end{tabular}
\paragraph{Preconditions:} 
\begin{itemize}  
\item  scriptLines != null
\item  scriptLines != undefined
\end{itemize}  
\paragraph{Returns:} \hfill \\ 
the input\\ 
\underline{Type:} String[]
\subsubsection{parse(script)} 
\paragraph{Description:} \hfill \\ 
Parses the input script to a script thar can be expanded by macros.
\paragraph{Parameters:} \hfill \\ 
\begin{tabular}{|l|l|l|}
\hline
\textbf{Name} & \textbf{Type} & \textbf{Description} \\ 
\hline
\texttt{script} & String & input (Accel) scrpit\\ 
\hline
\end{tabular}
\paragraph{Preconditions:} 
\begin{itemize}  
\item  script != null
\item  script != undefined
\end{itemize}  
\paragraph{Returns:} \hfill \\ 
exmpandable script.\\ 
\underline{Type:} String
\subsubsection{scriptToLines(script)} 
\paragraph{Description:} \hfill \\ 
Transforms the input {@code script} to an array of strings,
each representing a line.
\paragraph{Parameters:} \hfill \\ 
\begin{tabular}{|l|l|l|}
\hline
\textbf{Name} & \textbf{Type} & \textbf{Description} \\ 
\hline
\texttt{script} & String & the input script given as a String\\ 
\hline
\end{tabular}
\paragraph{Preconditions:} 
\begin{itemize}  
\item  script != null
\item  script != undefined
\end{itemize}  
\paragraph{Returns:} \hfill \\ 
an array containing all lines in {@code script}\\ 
\underline{Type:} String[]
\subsubsection{translateLine(a)} 
\paragraph{Description:} \hfill \\ 
Translates a line of Accel script to a line that can be expanded using macros.
Examples:
x = 5 becomes func(x = 5)
x = 5 ; kg becomes func(x = 5 ; {'kg' : 1})
z = 2 + sin(y + sin(x)) + 4 + sin(2) becomes func(z = 2 + sin(exe.y() + sin(exe.x())) + 4 + sin(2))
\paragraph{Parameters:} \hfill \\ 
\begin{tabular}{|l|l|l|}
\hline
\textbf{Name} & \textbf{Type} & \textbf{Description} \\ 
\hline
\texttt{a} & String & line of Accell script\\ 
\hline
\end{tabular}
\paragraph{Preconditions:} 
\begin{itemize}  
\item  line != null
\item  line != undefined
\end{itemize}  
\paragraph{Returns:} \hfill \\ 
Translated line\\ 
\underline{Type:} String
\subsubsection{translateRHS(rhs)} 
\paragraph{Description:} \hfill \\ 
Translates the right hand side of an Accel definition to a macro compatible string.
\paragraph{Parameters:} \hfill \\ 
\begin{tabular}{|l|l|l|}
\hline
\textbf{Name} & \textbf{Type} & \textbf{Description} \\ 
\hline
\texttt{rhs} & String & Right hand side of an Accel definitions\\ 
\hline
\end{tabular}
\paragraph{Preconditions:} 
\begin{itemize}  
\item  rhs != null
\item  rhs != undefined
\end{itemize}  
\paragraph{Returns:} \hfill \\ 
a macro compatible string.\\ 
\underline{Type:} String
\subsubsection{translateUnits(units)} 
\paragraph{Description:} \hfill \\ 
Translate the units of a certain line into our format, such
that it is an object in the executable code.
\paragraph{Parameters:} \hfill \\ 
\begin{tabular}{|l|l|l|}
\hline
\textbf{Name} & \textbf{Type} & \textbf{Description} \\ 
\hline
\texttt{units} & String & the units in a String format\\ 
\hline
\end{tabular}
\paragraph{Preconditions:} 
\begin{itemize}  
\item  units != null
\item  units != undefined
\end{itemize}  
\paragraph{Returns:} \hfill \\ 
a String to be used in our executable code.\\ 
\underline{Type:} String
\subsubsection{trimLines(scriptArray)} 
\paragraph{Description:} \hfill \\ 
Trims the lines in {@code scriptArray}.
\paragraph{Parameters:} \hfill \\ 
\begin{tabular}{|l|l|l|}
\hline
\textbf{Name} & \textbf{Type} & \textbf{Description} \\ 
\hline
\texttt{scriptArray} & String[] & the script formatted in a number of lines\\ 
\hline
\end{tabular}
\paragraph{Returns:} \hfill \\ 
the trimmed script in the format of a number of lines\\ 
\underline{Type:} String[]
\subsubsection{compile(sweet)} 
\paragraph{Description:} \hfill \\ 
Main and testable compilation function.
\paragraph{Parameters:} \hfill \\ 
\begin{tabular}{|l|l|l|}
\hline
\textbf{Name} & \textbf{Type} & \textbf{Description} \\ 
\hline
\texttt{sweet} &  & A possible reference to the sweet library.\\ 
\hline
\end{tabular}
\paragraph{Returns:} \hfill \\ 
Returns the number 6 if compilation and execution of the code with Sweet.js macro has all been succesful.\\ 
\section{Preprocessor} 
\textbf{Description: }First pass in compiling. Translates an Accel script to a script that can be expanded by macros.
\subsection{Members} 
\subsection{Methods} 
\subsubsection{parse(scriptlines)} 
\paragraph{Description:} \hfill \\ 
Parse the given lines of script.
Should be overridden by subclasses.
The base class only contains precondition checking and returns the given input.
\paragraph{Parameters:} \hfill \\ 
\begin{tabular}{|l|l|l|}
\hline
\textbf{Name} & \textbf{Type} & \textbf{Description} \\ 
\hline
\texttt{scriptlines} & String[] & Lines of script that need to be parsed.\\ 
\hline
\end{tabular}
\paragraph{Preconditions:} 
\begin{itemize}  
\item  scriptLines != null
\item  scriptLines != undefined
\end{itemize}  
\paragraph{Returns:} \hfill \\ 
the input\\ 
\underline{Type:} String[]
\subsubsection{parse(script)} 
\paragraph{Description:} \hfill \\ 
Parses the input script to a script thar can be expanded by macros.
\paragraph{Parameters:} \hfill \\ 
\begin{tabular}{|l|l|l|}
\hline
\textbf{Name} & \textbf{Type} & \textbf{Description} \\ 
\hline
\texttt{script} & String & input (Accel) scrpit\\ 
\hline
\end{tabular}
\paragraph{Preconditions:} 
\begin{itemize}  
\item  script != null
\item  script != undefined
\end{itemize}  
\paragraph{Returns:} \hfill \\ 
exmpandable script.\\ 
\underline{Type:} String
\subsubsection{scriptToLines(script)} 
\paragraph{Description:} \hfill \\ 
Transforms the input {@code script} to an array of strings,
each representing a line.
\paragraph{Parameters:} \hfill \\ 
\begin{tabular}{|l|l|l|}
\hline
\textbf{Name} & \textbf{Type} & \textbf{Description} \\ 
\hline
\texttt{script} & String & the input script given as a String\\ 
\hline
\end{tabular}
\paragraph{Preconditions:} 
\begin{itemize}  
\item  script != null
\item  script != undefined
\end{itemize}  
\paragraph{Returns:} \hfill \\ 
an array containing all lines in {@code script}\\ 
\underline{Type:} String[]
\subsubsection{translateLine(a)} 
\paragraph{Description:} \hfill \\ 
Translates a line of Accel script to a line that can be expanded using macros.
Examples:
x = 5 becomes func(x = 5)
x = 5 ; kg becomes func(x = 5 ; {'kg' : 1})
z = 2 + sin(y + sin(x)) + 4 + sin(2) becomes func(z = 2 + sin(exe.y() + sin(exe.x())) + 4 + sin(2))
\paragraph{Parameters:} \hfill \\ 
\begin{tabular}{|l|l|l|}
\hline
\textbf{Name} & \textbf{Type} & \textbf{Description} \\ 
\hline
\texttt{a} & String & line of Accell script\\ 
\hline
\end{tabular}
\paragraph{Preconditions:} 
\begin{itemize}  
\item  line != null
\item  line != undefined
\end{itemize}  
\paragraph{Returns:} \hfill \\ 
Translated line\\ 
\underline{Type:} String
\subsubsection{translateRHS(rhs)} 
\paragraph{Description:} \hfill \\ 
Translates the right hand side of an Accel definition to a macro compatible string.
\paragraph{Parameters:} \hfill \\ 
\begin{tabular}{|l|l|l|}
\hline
\textbf{Name} & \textbf{Type} & \textbf{Description} \\ 
\hline
\texttt{rhs} & String & Right hand side of an Accel definitions\\ 
\hline
\end{tabular}
\paragraph{Preconditions:} 
\begin{itemize}  
\item  rhs != null
\item  rhs != undefined
\end{itemize}  
\paragraph{Returns:} \hfill \\ 
a macro compatible string.\\ 
\underline{Type:} String
\subsubsection{translateUnits(units)} 
\paragraph{Description:} \hfill \\ 
Translate the units of a certain line into our format, such
that it is an object in the executable code.
\paragraph{Parameters:} \hfill \\ 
\begin{tabular}{|l|l|l|}
\hline
\textbf{Name} & \textbf{Type} & \textbf{Description} \\ 
\hline
\texttt{units} & String & the units in a String format\\ 
\hline
\end{tabular}
\paragraph{Preconditions:} 
\begin{itemize}  
\item  units != null
\item  units != undefined
\end{itemize}  
\paragraph{Returns:} \hfill \\ 
a String to be used in our executable code.\\ 
\underline{Type:} String
\subsubsection{trimLines(scriptArray)} 
\paragraph{Description:} \hfill \\ 
Trims the lines in {@code scriptArray}.
\paragraph{Parameters:} \hfill \\ 
\begin{tabular}{|l|l|l|}
\hline
\textbf{Name} & \textbf{Type} & \textbf{Description} \\ 
\hline
\texttt{scriptArray} & String[] & the script formatted in a number of lines\\ 
\hline
\end{tabular}
\paragraph{Returns:} \hfill \\ 
the trimmed script in the format of a number of lines\\ 
\underline{Type:} String[]
\subsubsection{compile(sweet)} 
\paragraph{Description:} \hfill \\ 
Main and testable compilation function.
\paragraph{Parameters:} \hfill \\ 
\begin{tabular}{|l|l|l|}
\hline
\textbf{Name} & \textbf{Type} & \textbf{Description} \\ 
\hline
\texttt{sweet} &  & A possible reference to the sweet library.\\ 
\hline
\end{tabular}
\paragraph{Returns:} \hfill \\ 
Returns the number 6 if compilation and execution of the code with Sweet.js macro has all been succesful.\\ 
\section{TemplateClass} 
\textbf{Description: }Classes can be defined as objects. Indiciate this using the @class param.
\subsection{Members} 
\subsection{Methods} 
\subsubsection{parse(scriptlines)} 
\paragraph{Description:} \hfill \\ 
Parse the given lines of script.
Should be overridden by subclasses.
The base class only contains precondition checking and returns the given input.
\paragraph{Parameters:} \hfill \\ 
\begin{tabular}{|l|l|l|}
\hline
\textbf{Name} & \textbf{Type} & \textbf{Description} \\ 
\hline
\texttt{scriptlines} & String[] & Lines of script that need to be parsed.\\ 
\hline
\end{tabular}
\paragraph{Preconditions:} 
\begin{itemize}  
\item  scriptLines != null
\item  scriptLines != undefined
\end{itemize}  
\paragraph{Returns:} \hfill \\ 
the input\\ 
\underline{Type:} String[]
\subsubsection{parse(script)} 
\paragraph{Description:} \hfill \\ 
Parses the input script to a script thar can be expanded by macros.
\paragraph{Parameters:} \hfill \\ 
\begin{tabular}{|l|l|l|}
\hline
\textbf{Name} & \textbf{Type} & \textbf{Description} \\ 
\hline
\texttt{script} & String & input (Accel) scrpit\\ 
\hline
\end{tabular}
\paragraph{Preconditions:} 
\begin{itemize}  
\item  script != null
\item  script != undefined
\end{itemize}  
\paragraph{Returns:} \hfill \\ 
exmpandable script.\\ 
\underline{Type:} String
\subsubsection{scriptToLines(script)} 
\paragraph{Description:} \hfill \\ 
Transforms the input {@code script} to an array of strings,
each representing a line.
\paragraph{Parameters:} \hfill \\ 
\begin{tabular}{|l|l|l|}
\hline
\textbf{Name} & \textbf{Type} & \textbf{Description} \\ 
\hline
\texttt{script} & String & the input script given as a String\\ 
\hline
\end{tabular}
\paragraph{Preconditions:} 
\begin{itemize}  
\item  script != null
\item  script != undefined
\end{itemize}  
\paragraph{Returns:} \hfill \\ 
an array containing all lines in {@code script}\\ 
\underline{Type:} String[]
\subsubsection{translateLine(a)} 
\paragraph{Description:} \hfill \\ 
Translates a line of Accel script to a line that can be expanded using macros.
Examples:
x = 5 becomes func(x = 5)
x = 5 ; kg becomes func(x = 5 ; {'kg' : 1})
z = 2 + sin(y + sin(x)) + 4 + sin(2) becomes func(z = 2 + sin(exe.y() + sin(exe.x())) + 4 + sin(2))
\paragraph{Parameters:} \hfill \\ 
\begin{tabular}{|l|l|l|}
\hline
\textbf{Name} & \textbf{Type} & \textbf{Description} \\ 
\hline
\texttt{a} & String & line of Accell script\\ 
\hline
\end{tabular}
\paragraph{Preconditions:} 
\begin{itemize}  
\item  line != null
\item  line != undefined
\end{itemize}  
\paragraph{Returns:} \hfill \\ 
Translated line\\ 
\underline{Type:} String
\subsubsection{translateRHS(rhs)} 
\paragraph{Description:} \hfill \\ 
Translates the right hand side of an Accel definition to a macro compatible string.
\paragraph{Parameters:} \hfill \\ 
\begin{tabular}{|l|l|l|}
\hline
\textbf{Name} & \textbf{Type} & \textbf{Description} \\ 
\hline
\texttt{rhs} & String & Right hand side of an Accel definitions\\ 
\hline
\end{tabular}
\paragraph{Preconditions:} 
\begin{itemize}  
\item  rhs != null
\item  rhs != undefined
\end{itemize}  
\paragraph{Returns:} \hfill \\ 
a macro compatible string.\\ 
\underline{Type:} String
\subsubsection{translateUnits(units)} 
\paragraph{Description:} \hfill \\ 
Translate the units of a certain line into our format, such
that it is an object in the executable code.
\paragraph{Parameters:} \hfill \\ 
\begin{tabular}{|l|l|l|}
\hline
\textbf{Name} & \textbf{Type} & \textbf{Description} \\ 
\hline
\texttt{units} & String & the units in a String format\\ 
\hline
\end{tabular}
\paragraph{Preconditions:} 
\begin{itemize}  
\item  units != null
\item  units != undefined
\end{itemize}  
\paragraph{Returns:} \hfill \\ 
a String to be used in our executable code.\\ 
\underline{Type:} String
\subsubsection{trimLines(scriptArray)} 
\paragraph{Description:} \hfill \\ 
Trims the lines in {@code scriptArray}.
\paragraph{Parameters:} \hfill \\ 
\begin{tabular}{|l|l|l|}
\hline
\textbf{Name} & \textbf{Type} & \textbf{Description} \\ 
\hline
\texttt{scriptArray} & String[] & the script formatted in a number of lines\\ 
\hline
\end{tabular}
\paragraph{Returns:} \hfill \\ 
the trimmed script in the format of a number of lines\\ 
\underline{Type:} String[]
\subsubsection{compile(sweet)} 
\paragraph{Description:} \hfill \\ 
Main and testable compilation function.
\paragraph{Parameters:} \hfill \\ 
\begin{tabular}{|l|l|l|}
\hline
\textbf{Name} & \textbf{Type} & \textbf{Description} \\ 
\hline
\texttt{sweet} &  & A possible reference to the sweet library.\\ 
\hline
\end{tabular}
\paragraph{Returns:} \hfill \\ 
Returns the number 6 if compilation and execution of the code with Sweet.js macro has all been succesful.\\ 
